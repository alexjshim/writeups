%        File: bernoulli.tex
%     Created: Sat Jun 05 06:00 PM 2021 P
% Last Change: Sat Jun 05 06:00 PM 2021 P
%
\documentclass[a4paper]{article}
\usepackage[]{amsmath}
\usepackage{amssymb}

\begin{document}

\section{Derivation 2 of Bernoulli distributions as an exponential family}
\indent Let $X$ be your sample space:
\begin{equation}
  X \in \{ 'heads', 'tails' \}
\end{equation}

Let us define a sufficient statistic, and induce the natural parameters.

We define the sufficient statistic by an indicator function

\begin{equation}
  u(x) = I( x = 'heads' )
\end{equation}

Note for multinoulli, we can specify indicator functions for $K-1$ classes, and obtain a sufficient statistic vector:

\begin{equation}
  \begin{split}
    u_1(x) &= I( x = 'cat' ) \\
    u_2(x) &= I( x = 'dog' ) \\
    u_3(x) &= I( x = 'hippo' ) \\
    \cdots
  \end{split}
\end{equation}

Now we define our intrinsic measure to restrict our samples to our discrete set of samples:

\begin{equation}
  h(x) = \begin{cases} 1\footnote{technically we would use a delta function $\delta(x = 'heads') + \delta(x = 'tails')$ for a continuous space} \quad x = 'heads' \, or \, x = 'tails' \\ 0 \quad \text{otherwise} \end{cases}
\end{equation}

The partition/normalization function can be calculated by integration/summing over X

\begin{equation}
  \begin{split}
    Z( \eta ) &= \sum_X e^{ u(x) \cdot \eta } \\
    &= e^{ 0 \cdot \eta } + e^{ 1 \cdot \eta} \\
    &= 1 + e^\eta
  \end{split}
\end{equation}

We know for exponential families

\begin{equation}
  \nabla_\eta \log Z(\eta) = E_{ x \sim p( x \vert \eta ) } \left[ u(x) \right]
\end{equation}

Using calculus to simplify the left side

\begin{equation}
  \begin{split}
    \frac{ d} { d \eta} \log ( 1 + e^\eta) &= \frac{ e^\eta }{ 1+ e^\eta } \\
    &= \sigma(\eta)
  \end{split}
\end{equation}

The right side is a sufficient statistic of the indicator function for heads

\begin{equation}
  E_{ x \sim p( x \vert \eta ) } \left[ I(x = 'heads' ) \right] = Pr\left( X = 'heads' \right)
\end{equation}

Thus, we have usual relation for bernoulli that we use for logistic regression

\begin{equation}
  Pr\left( X = 'heads' \right) = \sigma(\eta)
\end{equation}
\end{document}


